\documentclass[10pt]{article}


\usepackage[french]{babel}
\usepackage[T1]{fontenc}
\usepackage[utf8]{inputenc} % IL FAUT UTILISER L'ENCODAGE UTF8
\usepackage{verbatim}
\usepackage{graphicx}


\begin{document}

% -- Texte en italique/ gras --
%
% Le texte sous LateX peut �tre mis en italique, gras, soulign�, etc. Comme 
% sous word. Pour soumettre le texte � un certain style, on doit cependant
% utiliser la commande � cet effet et mettre le texte en param�tre � cette
% commande.
%
% Les tyles de texte principaux sont :
%
%	- textbf (Text Bold Face - gras)
%	- textit (Text Italic - italique)
%	- underline (souligne le texte, une alternative est la commande "emph")
%

\textbf{Bonjour, ceci est un exemple de texte en gras.}\\
\bigskip

\textit{Encore un exemple, cette fois-ci en italique!}\\
\bigskip

\underline{Autre exemple, soulign� cette fois.}\\
\bigskip

\textbf{\textit{Il est possible de combiner des styles au texte aussi.}};

\newpage

% -- Taille du Texte --
%
% En th�orie, LateX devrait toujours g�rer la taille du texte correctement pour
% vous. Si le texte est trop petit ou trop grand en g�n�ral, simplement modifier
% l'option de taille de police dans la classe du document devrait �tre suffisant.
%
% Dans les cas sp�ciaux o� vous souhaiteriez modifier malgr� tout la taille du
% texte, vous pouvez utiliser certaines commandes cr�es � cet effet. Puisque
% les tailles de texte sous LateX sont toujours relatives � la taille entr�e en
% option de classe du document, il n'est pas possible d'�crire directement 
% une taille de texte � respecter. Latex poss�de donc dix commandes de tailles
% diff�rentes qui g�n�rent une texte d'une certaine taille par rapport � la
% taille globalement configur�e.
%
% Du plus petit au plus grand on a donc :
%
% \tiny
% \scriptsize
% \footnotesize
% \small
% \normalsize
% \large
% \Large
% \LARGE
% \huge
% \Huge
%
\tiny{Ceci est un mini texte}

\small{Un peu plus gros.}

\normalsize{Taille normale.}

\Large{Plus gros}

\Huge{Le plus gros possible.}

% -- Couleurs de texte --
%
% Matlab ne supporte pas les couleurs nativement except� dans les images. Pour
% modifier la couleur du texte, vous devrez utiliser le package "color". Prenez
% note qu'il n'est cependant pas d'usage de mettre du texte en couleur dans
% un rapport scientifique. Il est donc conseill� d'�viter de le faire dans le 
% cadre de vos travaux pratiques.

\end{document}
